\documentclass{article}
\usepackage{CJKutf8}
\usepackage{latexsym}
\usepackage{pgfplots}  
\usepackage{pgfplotstable}

\title{Week 15}
\author{软73 沈冠霖 2017013569}
\begin{document}
\begin{CJK}{UTF8}{gkai}
\maketitle

\section{T1}
\paragraph{1}
思路:先初始化C让C全是0,然后每次分治先并行算出前四个乘积加到C上,再并行算出后四个乘积加到C上。伪代码如下。\\
1 n = A.rows\\
2 if n == 1\\
3 \qquad c11 += a11b11\\
4 else  partition A, B,C  into n/2 × n/2
submatrices: A11, A12, A21, A22;
B11, B12, B21, B22;C11,C12,C21,C22;respectively\\
5 \qquad spawn P-MATRIX-MULTIPLY-RE(C11, A11, B11)\\
6 \qquad spawn P-MATRIX-MULTIPLY-RE(C12, A11, B12)\\
7 \qquad spawn P-MATRIX-MULTIPLY-RE(C21, A21, B11)\\
8 \qquad P-MATRIX-MULTIPLY-RE(C22, A21, B12)\\
9  \qquad sync\\
10 \qquad spawn P-MATRIX-MULTIPLY-RE(C11, A12, B21)\\
11 \qquad spawn P-MATRIX-MULTIPLY-RE(C12, A12, B22)\\
12 \qquad spawn P-MATRIX-MULTIPLY-RE(C21, A22, B21)\\
13 \qquad P-MATRIX-MULTIPLY-RE(C22, A22, B22)\\
14 \qquad sync
\paragraph{2}
工作量:$T_{1}(n)=8T_{1}(\frac{n}{2})+\theta(1),T_{1}(1)=\theta(1)$\\。
持续时间:$T_{\infty}(n)=2T_{\infty}(\frac{n}{2})+\theta(1),T_{\infty}(1)=\theta(1)$\\
\paragraph{3}
根据主定理,工作量$T_{1}=\theta(n^{3})$,持续时间$T_{\infty}=\theta(n)$,并行度为$\theta(n^{2})$。\\
n=1000时,并行度为$10^{6}$,是先前算法$10^{7}$的十分之一。
\section{T2}
\paragraph{测试环境}
CPU:Inter Core i5-6300HQ,2.3GHZ\\
内存:12G\\
环境:VS2017,release模式
比较不同数据规模下串行和并行归并,快排的运行时间。两个都用的naive算法。\\
为了保证数据安全,我在并行的每个函数都开了全新的数组,复制元素过来进行排序,因此会自带$\theta(n)$的复杂度。
\paragraph{结果分析}
程序目前来看结果是正确的。\\
可以看出,使用多线程在递归上是非常慢的,并行归并比串行慢了成千上万倍,而且随着数据的增大,每增大2倍时间要增加10倍以上。我猜测是线程开的太多了,每个线程的算力太低了。并行快排比归并还慢很多,我猜测是线程切分不均匀导致的。
\begin{table}[!htbp] 
	
	\caption{不同数据规模下排序的时间}
	\begin{flushleft} 
		\begin{tabular}{|l|l|l|l|l|} 
			\hline 测量序号 & 1 & 2 & 3 & 4\\ 
			\hline 数据范围 &10&20&50&100 \\ 
			\hline 串行归并排序时间 (ms)
			&0&0&0.2&0 \\ 
			\hline 串行快速排序时间 (ms)
			&0&0&0&0\\ 
			\hline 并行归并排序时间 (ms)
			&1.2&42.8&1154.6&12872.8\\ 
			\hline 并行快速排序时间 (ms)
			&6&太大&太大&太大 \\ 
			\hline
		\end{tabular} 
		\\注:每组数据都是运行5次后取的平均值
	\end{flushleft} 
\end{table}

\pgfplotsset{}

\begin{tikzpicture}

\begin{axis}[legend pos=outer north east] % 将图例放在图外,位于图的东北角

\addplot 
table[]                         
{           		                
	N T
	10 0
	20 0
	50 0.2
	100 0

};
\addplot
table[] 
{   				
	N T
	10 0
	20 0
	50 0
	100 0	
};
\addplot
table[] 
{   				
	N T
	10 1.2
	20 42.8
	50 1154.6
	100 12872.8
};
\addplot
table[] 
{   				
	N T
	10 6
	20 15000
	50 15000
	100 15000
	
};

\addlegendentry{串行归并排序}
\addlegendentry{串行快速排序}
\addlegendentry{并行归并排序}
\addlegendentry{并行快速排序}

\end{axis}
\end{tikzpicture}\\
注:时间过大我用15s来表示了。


\end{CJK}
\end{document}
