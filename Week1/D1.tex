\documentclass{article}
\usepackage{CJKutf8}
\title{Week 1}
\author{软73 沈冠霖 2017013569}
\begin{document}
\begin{CJK}{UTF8}{gkai}
\maketitle
\section{T1} 
\paragraph{证明}
因为 $\exists c = 3,n_{0} = 1,$对于$\forall n >= n_{0},$有$cn^{2}>=2n$, 所以 2n 属于$ O(n^{2}) $ \\
因为$O(f(n))+\theta(f(n)) = \theta(f(n))$,则原式成立
\section{T2} 
\paragraph{证明}
若f(n)=$\theta(g(n))$,则$\exists c>0,$使得$n_{0}>0$,对于$\forall n>n_{0},f(n)>=cg(n)$。与f(n)=o(g(n))矛盾,因此所有f(n)=$\theta(g(n))$构成的集合和所有f(n)=o(g(n))构成的集合二者交集为空
\section{T3} 
\paragraph{证明}
取f(n) = sin($\pi$n/2)+1,g(n) = 1,则\\
$\exists c=2$,使得 $\exists n_{0} = 1,\forall n > n_{0},0<=f(n)<=cg(n)$,因此f(n)=O(g(n))\\
但是对于$\forall n_{0}>0$,总是$\exists n>n_{0},f(n) = 0,$,不存在c>0,使得$\exists n_{0},\forall n > n_{0},f(n)>=cg(n)$,因此$f(n) \neq \theta(g(n))$\\
同时$\exists c=0.1, \forall n_{0} > 0,\exists n > n_{0}$使得f(n) = 2,f(n)>=cg(n),因此 $f(n) \neq o(g(n))$\\
因此$\exists f(n)=O(g(n)), f(n) \neq o(g(n))$且$f(n) \neq \theta(g(n))$,原式成立\\
但是这里举出的反例在自然数定义域中不单调递增,对于一个算法复杂度来说没有实际意义\\
\section{T4} 
\paragraph{证明}
不妨令n=某一正实数$n_{t}$时,f(n)>=g(n),取c = 0.5,有$max(f(n_{t}),g(n_{t}))>=c(f(n_{t})+g(n_{t}))$,反之一样成立,则$\exists n_{0} = 1, \forall n_{t} > n_{0},max(f(n_{t}),g(n_{t}))>=c(f(n_{t})+g(n_{t}))$\\
而取c= 1,$\exists n_{0} = 1, \forall n > 0, f(n)+g(n)-max(f(n),g(n)) = min(f(n),g(n)) >= 0$\\
综上,$max(f(n),g(n))=\theta(f(n)+g(n))$
\section{T5} 
\paragraph{第一问}
由增长率由最小到最大可划分如下等价类:(每行一个等价类)\\
1,$n^{1/lgn}$\\
lg(lg*n)\\
lg*(lgn)\\
lg*(lgn)\\
lg*n\\
$2^{lg*n}$\\
ln(ln n)\\
$\sqrt{lgn}$\\
ln n\\
$(lg n)^{2}$\\
$(\sqrt{2})^{lgn}\\
n,2^{lgn}\\
nlgn,lg(n!)\\
n^{2},4^{lgn}\\
n^{3}\\
2^{\sqrt{2lgn}}\\
(1.5)^{n}\\
2^{n}\\
e^{n}\\
n*2^{n}\\
n^{lg lgn},(lg n)!,(lgn)^{lgn}\\
n!\\
(n+1)!\\
2^{2^{n}}\\
2^{2^{n+1}}$\\
方法:对于阶乘,幂取对数处理,对于对数,取指数处理。应用结论$lg(n!)=\theta(nlgn)$\\
\paragraph{第二问}
$f(n) = 2^{2^{2n}} + (-1)^{n}2^{2^{2n}}$,n为正整数
\end{CJK}
\end{document}
