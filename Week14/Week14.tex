\documentclass{article}
\usepackage{CJKutf8}
\usepackage{latexsym}
\usepackage{pgfplots}  
\usepackage{pgfplotstable}

\title{Week 14}
\author{软73 沈冠霖 2017013569}
\begin{document}
\begin{CJK}{UTF8}{gkai}
\maketitle
\section{T1}
先把集合覆盖问题化简为如下形式:给定集合S和若干个子集$S_{1},S_{2}....S_{m}$和常数k,是否存在k个子集,使其并为S。\\
1.如果给定了这k个子集,则只需要把它们并起来就可以验证了,时间复杂度O(n)。因此集合覆盖是NP问题。\\
2.给定一个顶点覆盖问题的图G和值k,可以找到一个函数$f(G,k)=(S,S_{1},S_{2}....S_{m},k')$(以下把集合覆盖问题简写为(S,k)),使得k'=k,S为所有边构成的集合,子集$S_{1},S_{2}...S_{m}$分别对应图G的顶点1到m关联的边集。这个函数能把任意的顶点覆盖问题映射为集合覆盖问题。\\
3.如果顶点覆盖问题(G,k)成立,那么可以找到k个顶点,使得每条边的起点或终点都在这k个顶点中找到。那么对应的集合覆盖问题,集合中每个元素都可以在这k个顶点对应的集合中找到,对应的集合覆盖问题f(G,k)=(S,k)也成立。反之,如果集合覆盖问题(S,k)=f(G,k)成立,那么必定存在k个集合,使得全集S中任何一个元素都能在这k个子集中被找到,那么对于顶点覆盖问题(G,k),任何一个边都能在这k个子集对应的顶点的边集中找到,也就是这k个顶点覆盖了整个图,顶点覆盖问题(G,k)也成立。\\
4.算法只需要复制k,同时统计每条边,把每条边加入全集和对应的点子集就可以了,时间复杂度$\theta(E)$,是多项式算法。
\section{T2}
\paragraph{算法}
1: 遍历每个子集,来初始化如下数据结构:R[size]:每个元素R[i]是一个链表,存储着剩余元素(未被覆盖元素)为i个的集合,size为子集中元素最多数目。R[i]的元素是集合S,S.key是这个集合的剩余元素个数。A[n]:每个元素是一个链表,存储着每个元素位于哪些集合。visit[n]:存储每个元素是否被覆盖。ans:存储被覆盖的元素个数。max:存储当前子集剩余元素最多是几。\\
2:之后进行k次循环,步骤如下:\\
2.1:提取并删除R[max]的表尾集合$S_{max}$(一会要扩增这个集合),将ans加上max,如果ans=n了就结束,如果R[max]空了就更新max。\\
2.2:遍历集合$S_{max}$的每个元素$a_{i}$,如果$visit[a_{i}]=0$,就更新visit,同时遍历$A[a_{i}]$,把得到的每个集合,也就是用到这个元素的每个集合的key都-1,从R表的对应位置R[key]删除,并且加入R[key-1]的尾。
\paragraph{证明}
首先,每次提取的都一定是剩余元素最多(可以扩增元素最多)的集合,算法正确。\\
其次,分析其复杂度。初始化需要遍历每个子集的所有元素,复杂度是$O(\Sigma|S|)$。考虑所有循环结束后的所有操作数。首先,提取最大的集合提取了k次,每次都是O(1),因此总共是O(k)。其次,max从size减小到了0,更新max的复杂度是O(size)。最后,每次遍历需要遍历每个扩增元素的所有对应子集,整套遍历之后相当于遍历了每个子集的每个元素各一次,遍历了$O(\Sigma|S|)$次。而每遍历每个元素对应的子集,操作链表的复杂度是O(1)。因此遍历和更新复杂度是$O(\Sigma|S|)$。总复杂度是$O(\Sigma|S|+k+size)=O(\Sigma|S|)$。
\end{CJK}
\end{document}
