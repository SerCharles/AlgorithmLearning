\documentclass{article}
\usepackage{CJKutf8}
\usepackage{latexsym}
\usepackage{pgfplots}  
\usepackage{pgfplotstable}

\title{Week 4}
\author{软73 沈冠霖 2017013569}
\begin{document}
\begin{CJK}{UTF8}{gkai}
\maketitle

\section{T1} 
证明:设$0 \leq \alpha \leq 0.5$,且n足够大,则设三次取的数分别为a,b,c,要求最坏为$\alpha, 1-\alpha$的分割的概率,也就是求中位数落在$(n\alpha,n - n\alpha)$间的概率。\\
先考虑一种情况:$a\leq b \leq c \cap n\alpha \leq b \leq n - n\alpha$\\
设$P(a\leq b \leq c | n\alpha \leq b \leq n - n\alpha) = \int_{\alpha}^{1-\alpha}x(1-x)dx$ = P1\\
$P(n\alpha \leq b \leq n - n\alpha) = 1-2\alpha = P2$\\
根据条件概率公式,所求概率 = $P1*P2$\\
而a,b,c的大小排列有六种互斥,等概率的情况,所以总概率为 $6P1*P2 = 4\alpha^{3}-6\alpha^{2}+1$\\
\section{T2}
\paragraph{1}
证明:
假设存在常数$c_{1},c_{2}$,有对于$\forall n, c_{1}n^{2} \leq T(n) \leq c_{2}n^{2}$,而T(1)=1,必定存在这两个常数使得T(1)满足。\\
设对于n = k满足条件,则对于n = k+1,$T(k+1) = T(k)+\theta(n)=T(k)+c(k+1)+d$,而$c_{1}k^{2} \leq T(k) \leq  c_{2}k^{2}$,只需要满足$2c_{1}k \leq ck+(c+d-1) \leq 2c_{2}k$即可,当k足够大时,取$c_{1} \leq \frac{c}{2} <c_{2}$即可 
\paragraph{1}
每次划分都返回r,相当于左边都是长n-1的段,右面无效段,因此有$T(n) = T(n-1) + \theta(n)$\\
假设存在常数$c_{1},c_{2}$,有对于$\forall n, c_{1}n^{2} \leq T(n) \leq c_{2}n^{2}$,而T(1)=1,必定存在这两个常数使得T(1)满足。\\
设对于n = k满足条件,则对于n = k+1,$T(k+1) = T(k)+\theta(n)=T(k)+c(k+1)+d$,而$c_{1}k^{2} \leq T(k) \leq  c_{2}k^{2}$,只需要满足$2c_{1}k \leq ck+(c+d-1) \leq 2c_{2}k$即可,当k足够大时,取$c_{1} \leq \frac{c}{2} <c_{2}$即可\\
因此,$T(n) = \theta(n^{2})$
\paragraph{2}

1\qquad PARTITION'(A,p,r)\\
2\qquad x = A[r]\\
3\qquad i = p - 1\\
4\qquad q = p - 1\\
5\qquad for j = p  to  r-1\\
6\qquad\qquad if A[j] $<$ x\\
7\qquad\qquad\qquad  i = i + 1\\
8\qquad\qquad\qquad exchange A[i] with A[j]\\
9\qquad\qquad\qquad q = q + 1\\
10\qquad\qquad\qquad exchange A[i] with A[q]\\
11\qquad\qquad else if A[j] <= x\\
12\qquad\qquad\qquad i = i + 1\\
13\qquad\qquad\qquad exchange A[i] with A[j]\\
14\qquad exchange A[i+1] with A[r]\\
15\qquad t = i + 1\\
15\qquad q = q + 1\\
16\qquad return q,t\\
\paragraph{3}

1\qquad RANDOMIZED$\_$PARTITION'(A,p,r)\\
2\qquad i = RANDOM(p,r)\\
3\qquad exchange A[r] with A[i]\\
4\qquad return PARTITION'(A,p,r)\\
\\
1\qquad QUICKSORT'(A,p,r)\\
2\qquad if(p$<$r) then\\
3\qquad (q,t)=RANDOMIZED$\_$PARTITION'(A,p,r)\\
4\qquad QUICKSORT'(A,p,q-1)\\
5\qquad QUICKSORT'(A,t+1,r)\\


\paragraph{4}
如果A[i] $!=$A[j],那么这两个元素比较过的概率还是$\frac{2}{j-i+1}$\\
而如果两个数相等,假设这个数出现了k次,则这两个元素比较过的概率是$\frac{k-1}{C_{K}^{2}}=\frac{2}{k}$,而因为$j-i+1\leq k$,因此新的比较次数小于等于原来的比较次数,因此结论不变

\end{CJK}
\end{document}
