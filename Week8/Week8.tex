\documentclass{article}
\usepackage{CJKutf8}
\title{Week 7}
\author{软73 沈冠霖 2017013569}
\begin{document}
\begin{CJK}{UTF8}{gkai}
\maketitle
\section{T1} 
\subsection{1}
\paragraph{算法}
把每个任务的执行时间由小到大排序,然后对排好序的任务序列从前往后执行
\paragraph{证明}
设第i个执行完的任务是第$t_{i}$个任务,那么其完成时间是$c_{i} = \sum_{j = 1}^{i}p_{t_{i}}$,总共的完成时间是$\sum_{i=1}^{n} c_{i} = \sum_{i=1}^{n}(n+1-i)p_{t_{i}}$。\\
而假设使得总完成时间最短的任务执行序列中存在i,j,使得$i<j,p_{t_{i}}>p_{t_{j}}$,则将$t_{i},t_{j}$两个任务交换顺序,总完成时间会减少$(j-i)(p_{t_{i}}-p_{t_{j}})$。因此序列$p_{t_{i}}$必须单调递增。
因为n是常数,所以总共完成时间最短,则平均完成时间最短。因此算法正确。
\paragraph{复杂度}
先对任务的执行时间排序,复杂度O(nlgn),之后依次求出每个任务的完成时间,并且求和取平均,复杂度$\theta(n)$,因此总共时间复杂度O(nlgn)。\\
\subsection{2}
\paragraph{算法}
维护一个优先级队列q,里面存储当前时间t时,可以开始进行的任务的剩余完成时间,剩余完成时间越短,优先级越高。\\
每次操作把时间t加一,然后在新的时间t可以开工的任务加入队列,然后执行队列头的任务。如果执行完毕,这个任务弹出队列,并且记录其结束时间。直到队列为空停止,计算平均完成时间。
\paragraph{证明}
1.证明是最优子结构:设每个状态t[$i,j_{1},j_{2},...j_{n}$]记录时间i后每个任务的剩余执行时间,每个时间做出选择,选择哪个可执行任务在这个时间执行。则每个选择的最优解必定是这次选择后完成的总时间+这个选择下一个状态后完成的最优总时间。\\
2.证明贪心选择正确:设时间t时,可以执行的剩余执行时间最短的任务是j,而最优选择在这一时间却选择了执行时间$r_{i}>r_{j}$的任务i。设在最优选择中执行任务i,j的时间分别构成数列$t_{i},t_{j}$,完成时间分别为$e_{i},e_{j}$。\\
先证明最优情况必定有$e_{j}<e_{i}$。如果最优情况是$e_{j}>e_{i}$,则在不改变其他任务执行状况前提下,无论怎么分配i,j的任务执行时间,任务i,j全都执行完的时间必定是$e_{j}$。而如果选取$t_{i},t_{j}$合并成的数列的前$p_{j}$个时间执行j,剩下时间执行i,这样i,j的任务结束总时间一定是最小的,因此一定有$e_{j}<e_{i}$\\
再证明如果$e_{j}<e_{i}$,则如果这个时间选择第j个任务,这个选择一定包含在某个最优解里。设某个最优解在这个时间执行的是第i个任务,那么一定有$e_{j}<e_{i}$,那么如果把这个时间换成执行第j个任务,在原先的$e_{j}$时间执行第i个任务,那么$e_{i}$不变,$e_{j}$必定变小,其他任务完成时间不变,总完成时间变小。说明某个最优解在这个时间一定执行的是第j个任务。\\
\paragraph{复杂度}
假设在时间i有$m_{i}$个任务被推入队列,最多一个任务被推出队列,那么每个时间花费$O((m_{i}+1)lgn)$,因为$\sum_{t=1}^{max(t)}m_{t}=n$,则总共时间复杂度为O((n+T)lgn),T为最后一个任务完成的时间。因为每个任务时长至少为1,而且因为开始时间的限制,可能有的时间并未执行任务,所以n=O(T),总共时间复杂度为O(Tlgn)。
\section{T2}

\paragraph{证明}
首先,缓存不是命中就是未命中,因此最大命中次数等价于最小未命中次数。\\
1.证明是最优子结构:设每个状态$f[i,j]$,代表读取i个缓存数据后,而且缓存中的数据为大小小于等于k的集合j时的最小缓存不命中次数。每次选择如下:i增加1,如果新的缓存数据命中则无序集合j不变,否则j先加入未命中的数据,再移除至多一个数据使得集合j大小小于等于k。因为这次做的选择并不影响从新的状态到结束的最优解,因此假设$f[i,j]$为从输入完毕第i个数据后开始,初始缓存集合为j,直到输入完毕结束的最小缓存不命中次数,第一个选择使得j变成j',那么必定有$f[i,j]|choose_1=g(choose\_1)+f[i+1,j']$,其中$g(choose\_1)$代表第一个选择中未命中的数据个数,$f[i,j]|choose\_1$代表做了这一个选择前提下的最优值。因此有最优子结构。\\
2.证明贪心可以进行:假设要求任意的$f[i_{m},j_{m}]$,假设第$i_{m}+1$次输入未命中而且缓存满了,把未命中的数据加入缓存。\\
假设使得$f[i_{m},j_{m}]$取得最优解的第一个选择是移除数据i,而实际上可以移除的,最远出现的数据是j。假设i,j下次出现分别在$c_{i},c_{j}$个元素后,$c_{i}<c_{j}$,分四种情况讨论。在每种情况中,我提出的新方法都是第一次操作移除j保留i,最优解都是移除i保留j,而两种方法中任何和元素i,j无关的操作都完全一致。图像解释参见图片1-4\\
情况1,如果最优解直到输入结束都没让i,j进出缓冲区,假设到结束一共a个i,b个j,到第一个j之前一共有d个i,其余未命中缓冲区c次,$0\leq a,b,c;1\leq d$。则可以不改变其余的进出缓冲区操作,一开始移除j保留i,到了第一个j之后移除i保留j,其余情况不对i,j做任何进出缓冲区操作,那么原先最优解会未命中$a+1$次,而新的情况会命中$a-d+2\leq a+1$次。\\
而如果遇到特殊情况,再也不会遇到j了,那么假设之后会遇到a个i,其余未命中c次,$0\leq a,c$,那么原最优解未命中$a+c$次,新的情况则为不进行任何关于i,j的缓冲区操作,未命中$c\leq a+c$次。\\
情况2,如果最优解第一次修改缓冲区的i,j操作是让i进入缓冲区,j出缓冲区,则这个输入的字符必定是i,假设在遇到第一个j前就交换了i与j,那么交换之前有d个i,其余有c次未中,$0\leq d,c$,那么最优解需要交接$d+c+1$次。而新的情况是不进行i或j的进出缓冲区,这样能命中所有的i,只需要交换$c\leq d+c+1$次,而他们在最优解交换i与j后的状态完全相同。\\
假设在遇到第一个j后交换了i与j,假设第一个j前有d个i,第一次交换i与j的前面有a个i,b个j,其余交换c次,$1\leq d\leq a,1\leq b,0\leq c$,那么最优解需要交换$c+a+1$次。而新的情况是先在第一个j处交换i和j,然后再在最优解第一次交换的位置交换i与j。这样只需要交换$a-d+2+c\leq c+a+1$次,而且交换完毕后与最优解得到的状态完全相同。\\
情况3,如果最优解第一次修改缓冲区的i,j操作是将j换出缓冲区,换上不是i也不是j的元素t。先假设这次修改在遇到第一个j后进行,修改前遇到了a个i,b个j,第一个j前有d个i,其余修改c次,$1\leq d\leq a, 1\leq b, 0 \leq c$,那么最优解未命中了a+1+c次。而新的操作是在第一个j位置换上j换下i,之后在最优解的第一次交换位置换下j换上t。这样只未命中$a-d+2+c\leq a+1+c$次,而且交换完毕后和最优解得到的状态完全相同。\\
而假设这次修改在遇到第一个j前就进行,修改前遇到了a个i,其余修改了c次,$0\leq a,c$,那么最优解一共未命中a+1+c次。而新的做法是不动,只在原来最优解第一次修改的时候换下i换上t,那么一共只修改了$1+c\leq a+1+c$次,而且状态和最优解得到的完全相同。\\
情况4,假设最优解第一次修改是把i加入缓冲区,换下t。假设这次操作在遇到第一个j后进行,那么假设进行这次操作前遇到了a个i,b个j,第一个j前遇到了d个i,其余未命中操作为c,$1\leq d\leq a,1\leq b,0\leq c$,那么一共未命中$a+1+c$次。而更新的操作是第一次遇到j的时候先换上j换下i,在和最优解第一次操作同时换上i换下t。这样一共未命中$a-d+2+c\leq a+1+c$次,而且结束的状态和最优解完全相同。\\
假设最优解第一次操作在遇到第一个j前进行,假设进行前遇到了a个i,进行后到第一个j遇到了b个i,最优解其余未命中c次,$0\leq a,b,c$,那么等到遇到第一个j时一共未命中a+1+c次。而更新操作是先什么都不做,遇到第一个j时换上j换下t,这样只未命中$1+c'$次,而且因为缓冲区里有t,因此遇到t都能命中,$c'\leq c$,因此$1+c'\leq a+c+1$,而且最后状态和最优解完全相同。\\
综上,无论是哪种情形,第一次选择移除j总能得到某个最优解,因此贪心正确。
\paragraph{代码和复杂度}
为了方便,假设一开始缓存就是满的\\
情况1:遇到的元素种类较多,而元素个数没那么多\\
1\qquad MANAGE$\_$BUFFER(L,n,k,B)\\
2\qquad for i=1 to n\\
3\qquad\qquad success = 0\\
4\qquad\qquad for j=1 to k\\
5\qquad\qquad\qquad if L[i]=B[j]\\
6\qquad\qquad\qquad\qquad success = 1\\
7\qquad\qquad\qquad\qquad break\\
8\qquad\qquad if success = 0\\
9\qquad\qquad \qquad for ii = 1 to k\\
10\qquad\qquad\qquad\qquad next[ii] = n-i+1\\
11\qquad\qquad\qquad\qquad for jj = 1 to n-i\\
12\qquad\qquad\qquad\qquad\qquad if B[ii] = L[i+jj]\\
13\qquad\qquad\qquad\qquad\qquad\qquad next[ii]=jj\\
14\qquad\qquad\qquad\qquad\qquad\qquad break\\
//找到使得next最大的数组下标\\
15\qquad\qquad\qquad maximum = argmax(next)\\
\\
//输出处理第几个请求的时候逐出了哪个元素\\
16\qquad\qquad\qquad print(i,B[maximum])\\
17\qquad\qquad\qquad B[maximum]=L[i]\\
18\qquad return \\

情况2:遇到的元素种类有限,仅有m种,而元素数量很多\\
1\qquad MANAGE$\_$BUFFER(L,n,k,B,m)\\
2\qquad for i=1 to n\\
3\qquad\qquad queue[L[i]].push(i)\\
4\qquad for i=1 to m\\
5\qquad\qquad queue[i].push(n+1)\\
6\qquad for i=1 to k\\
7\qquad\qquad whether$\_$inside[B[i]]=1\\
8\qquad for i=1 to n\\
9\qquad\qquad if whether$\_$inside[L[i]]=1\\
10\qquad\qquad\qquad continue\\
11\qquad\qquad else\\
12\qquad\qquad\qquad $max\_next = 0$\\
13\qquad\qquad\qquad for j=1 to k\\
14\qquad\qquad\qquad\qquad if queue[B[j]].first > $max\_next$\\
15\qquad\qquad\qquad\qquad\qquad max$\_$next = queue[B[j]].pop()\\
16\qquad\qquad\qquad\qquad\qquad max$\_place$ = j\\
17\qquad\qquad\qquad whether$\_$inside[B[$max\_place$]]=0\\
18\qquad\qquad\qquad whether$\_$inside[L[i]]=1\\
19\qquad\qquad\qquad B[$max\_place$] = L[i]\\
20\qquad\qquad\qquad print(i,B[$max\_place$])\\
21\qquad return \\

\paragraph{复杂度}
第一种情况:最坏情况:每次缓存都不命中,则每输入一个元素,都需要用k次循环判断是否命中,再用k(n-i)次找每个元素下次在哪里遇到,再用k次找到哪个元素该出去,一共需要$\sum_{i=1}^{n}k(n+2-i) = O(kn^{2})$的时间复杂度,并且需要额外空间O(k)来存储每个缓冲区元素的下一个位置。\\
第二种情况:最坏情况同上,预处理需要O(m+n+k)的复杂度,而循环的时候每次只需要O(nk)的复杂度来找移除哪个元素就可以了,一共需要O(nk+m)的时间复杂度。需要m个总长度n+m的队列来存储每个元素出现的位置,以及m的表来存储每个元素在不在缓冲区。O(m+n)的空间复杂度。\\

\end{CJK}
\end{document}
