\documentclass{article}
\usepackage{CJKutf8}
\usepackage{latexsym}
\usepackage{pgfplots}  
\usepackage{pgfplotstable}

\title{Week 12}
\author{软73 沈冠霖 2017013569}
\begin{document}
\begin{CJK}{UTF8}{gkai}
\maketitle
\section{T1}
证明:如果有$L_{1}\leq pL_{2},L_{2}\leq pL_{3}$,则必定可以找到函数$f(x),g(x)$,使得$\forall x, x\in L_{1} \leftrightarrow f(x)\in L_{2}$,$\forall x, x\in L_{2} \leftrightarrow g(x)\in L_{3}$。则任取一个$x_{0}\in L_{1}$,必定可以找到$y_{0}=f(x_{0})\in L_{2},z_{0}=g_(y_{0})\in L_{3}$,也就有$g(f(x_{0}))=z_{0}$。同时,任取一个$z_{1}\in L_{3},$必定可以找到$y_{1}\in L_{2},g(y_{1})=z_{1},x_{1}\in L_{1},f(x_{1})=y_{1}$,也就有$g(f(x_{1}))=z_{1}$。这样,取函数h(x)=g(f(x)),就有$\forall x,x\in L_{1}\leftrightarrow g(f(x))]\in L_{3}$。

\section{T2}
证明:这个问题应该转化为判定问题:给定图G和值k,是否存在一个G中简单回路总长度为k。\\
1.给定一个图X和一个代表回路顶点序列的串Y,只需要遍历Y,如果能构成长度为k的简单回路则成功,否则如果不是回路,或者不是简单回路,或者长度不是k,则失败。算法A(X,Y)=1复杂度为O(n)。\\
2.用哈密顿回路问题进行归约。定义函数$f(G)=(G,|V|)$(如果这个问题不带权,那么就这样了。否则把每条边权值设为1)。首先,如果G满足存在哈密顿回路,则这个图中必定存在一个简单回路-这个哈密顿回路,长度为$|V|$。反之,如果G不满足存在哈密顿回路,则一定不存在一个长度为$|V|$的简单回路。因此成功构造了满足要求的函数f。\\
3.f只需要把G和$|V|$复制过去,复杂度是O(n),甚至O(1),是一个多项式复杂度算法。
\section{T3}
\paragraph{a}
定义:给定图G和非负整数k,是否存在一个大小为k的集合$V' \subset V$,使得$V'$满足独立性。\\
证明:1.可以找到一个判定算法A(X,Y),令X为给定的图G和非负整数k,Y'为这个图大小为k的子点集,只需要遍历G中所有边,如果每条边都最多与Y'最多一个点相关,则成立。否则不成立。遍历时间复杂度为$\theta(m)=O(n^{2})$,每次遍历判断2次(常数次),因此算法$A=O(n^{2})$,是多项式算法,$L\in NP$。\\
2.用NPC问题--团问题进行归约。定义函数$f(G,k)=(\overline{G},k)$。而如果图G和值k满足团问题,也就是能找到一个子点集V',使得V'中任意两个不同的点之间都有边,则对于同一个V',在$\overline{G}$中,任意两个不同的点之间都没有边,也就是不存在$\overline{G}$中的一条边,其关联着V'中两个不同节点,也就是V'是$\overline{G}$大小为k的独立集。反之,如果存在一个子点集V'是$\overline{G}$大小为k的独立集,则在$\overline{G}$中,V'间任意两个不同的点都没有边,则G中V'间任意两个不同的点都有边,(G,k)满足团问题条件。因此找到了这样一个正确的函数f(G,k),把团问题归约到独立集问题。
3.f(G,k)的操作是把邻接矩阵的每个值取非,时间复杂度是$\theta(n^{2})$,因此是多项式复杂度算法。
\paragraph{b}
因为子集V'大小最小为0,最大为$|V|$,因此从k=$|V|$开始,由大到小遍历$|V|$到0之间所有整数,如果找到第一个成功的$k_{0}$就停止。时间复杂度为$\theta(|V|)$。\\
\paragraph{c}
1.因为每个点度数为2,因此一共有n条边。而且连通分支的个数和回路个数相等。\\
2.每个连通分支都是一整个回路。因为在每个连通分支中,每个点度数是2,因此每个连通分支的边数和点数一样,因此有且仅有一个回路。而因为每个点度数是2,如果这个连通分支不是一个回路的话,必定存在一个点,其度数为1,矛盾。\\
3.因此我的算法如下:遍历每个回路,也就是连通分支,如果回路长是1,就把这个点加入集合。如果回路长是偶数,也就是2n,就每隔一个节点把节点加入集合,一共加入n个节点。如果回路长是奇数,也就是2n+1,就每隔一个节点把节点加入集合,一共加入n个节点。根据回路的性质,这个算法正确,复杂度为$\theta(m)=\theta(n)$。\\
\paragraph{d}
1.先求出最大匹配大小为$|M|$,则不在最大匹配对中的节点有$|V|-2|M|$个,把这些加入独立集合S内。因为这些节点一定互相没有边,否则构成新的匹配,与最大匹配矛盾。\\
2.每个最大匹配对中必定可以找到一个节点加入S中。如果第i个最大匹配对中,两个节点a,b都不可以加入S,则必定有边(a,b),(a,x),(b,y),而且$x,y\in S$。那么$(a,x),(b,y)\notin M,(a,b)\in M$,构成了一条增广路径,与最大匹配没有增广路径矛盾。因此至少一个节点可以加入S。\\
3.这样找到的是最大匹配:因为这样找到的$|S|=|V|-|M|$,如果有比这个还大的集合,则根据鸽巢原理,必定有集合中两个点属于同一个匹配对,则这两个点间有边,矛盾。\\
4.因此我的算法如下:先按照26.3的算法求出最大匹配,然后把不在最大匹配里的点全部加入S,之后遍历每个最大匹配对,在每个匹配对里找到一个能加入集合S的点。求最大匹配的算法复杂度O(VE),匹配对有O(V)个,在每个匹配对里加入一个节点最多遍历E条边,因此总共复杂度为O(VE)。

\end{CJK}
\end{document}
