\documentclass{article}
\usepackage{CJKutf8}
\title{Week 9}
\author{软73 沈冠霖 2017013569}
\begin{document}
\begin{CJK}{UTF8}{gkai}
\maketitle
\section{T1} 
证明:假设删除一个元素没有导致size改变,那么$c_{i}=1$,$\hat{c_{i}}=c_{i}+\phi_{i}-\phi_{i-1}=1+|2T.num_{i}-T.size_{i}|-|2T.num_{i-1}-T.size_{i-1}|$,而$T.size_{i} = T.size_{i-1},T.num_{i}=T.num_{i-1}-1$,因此原式可化为\\
$\hat{c_{i}}=1+|2T.num_{i}-T.size_{i}|-|2T.num_{i}-T.size_{i}+2|\leq 3$\\
而如果删除一个元素导致size改变了,那么有$c_{i}=1+T.num{i},T.size_{i} = \frac{2}{3}T.size_{i-1},T.num_{i}=T.num_{i-1}-1,T.num_{i-1}=\frac{1}{3}T.size_{i-1}$,原式可化为\\
$\hat{c_{i}}=1+T.num{i}-2T.num_{i}+T.size_{i}+2T.num_{i-1}-T.size_{i-1}\\
=3+\frac{1}{3}T.size_{i-1}-1+\frac{2}{3}T.size_{i-1}-T.size_{i-1}=2$
\section{T2} 
\subsection{1}
最坏情况是每个数组都是满的,搜索没个数组的时间复杂度是$O(i)=O(lgn)$,一共$\lceil lg(n+1) \rceil=O(lgn)$个数组,最坏时间复杂度是$O((lgn)^{2})$\\
\subsection{2}
\paragraph{算法}
从m-1个数插入一个数使得数组变为m个数的时候,找到最小的二进制位号k使得二进制代表$2^{k-1}$的一位从0变1,那么前面k-1位一定都是从1变0。\\
把前面k-1个数组和新的数一起做归并锦标排序(每个数组待更新的第一个数构成一个败者树,每次更新把败者树根节点推入第k个数组中,之后更新败者树。直到第k个数组被排满为止\\
\paragraph{复杂度}
因为败者树一共m个叶子节点,初始建树代价O(m),每次更新代价O(lgm),一共需要更新$2^{m}$次,因此时间复杂度$O(lgm2^{m})$。而最坏情况是第$\lceil lg(n+1) \rceil$位从0变成1,时间复杂度就是O(nlglgn)\\
而采用聚集法进行均摊分析,令$k=\lceil lg(n+1) \rceil$,则$2^{k-1}\leq n <2^{k}$,那么$m=k-1$最多有1次,$m=k-2$最多有2次,以此类推,$m=i$最多有$2^{k-i-1}$次。对前m-1个数组和新加的数做归并排序进入第m个数组的时间复杂度是$O(lgm2^{m})$,因此对长度为0到n-1的数组插入1个数的总共时间复杂度求和,总复杂度为$\sum_{i=0}^{k-1}2^{k-1}lgi =k2^{k-1}lg(k-1)!=O(2^{k-1}k^{2}lgk)=O(n(lgn)^{2}lglgn) $,均摊后每次插入时间复杂度是$O((lgn)^{2}lglgn)$
\subsection{3}
\paragraph{算法}
从长为m的数组删除一个数的时候,也必定可以找到最小的二进制位号k使得二进制代表$2^{k}$的一位从1变0,那么前面k-1位一定都是从0变1。\\
先把待删除的数a插入第k个数组中,也就是用a替换大于等于a的最小数b。之后把b插入a原来所在的数组。(如果a原来就在第k个数组中,则不必操作)\\
之后把第k个数组按照顺序依次写入前k-1个数组里。\\
\paragraph{复杂度}
先考虑对于任意一个长度m删除一个数的最坏情况:这些数组满足一个条件:每个数组内部由小到大排序,但是标号小的数组中任何一个数都比标号大的数组中任一个数大。而且每次都是删除标号最大的满数组中的一个数,设这个最大标号是$i_{max}$,而二进制由1变0的最小标号是$i_{min}$。把待删除的数m放到标号$i_{min}$的数组里需要$O(i_{min})$的时间,而把替换得到的数换到标号为$i_{max}$的数组则需要$O(2^{i_{max}})$的时间。而把标号为$i_{min}$的数组分到前$i_{min}$个数组中需要时间$O(2^{i_{min}})$。\\
对于长度1到n的最坏情况是$i_{max}=\lceil lg(n+1) \rceil,i_{min}=\lceil lg(n+1) \rceil-1$,此时最坏时间复杂度是O(n)。\\
先均摊$i_{min}$和把标号为$i_{min}$的数组分到前$i_{min}$个数组中的时间,使用聚集法进行均摊。长度为1到n中,$k=\lceil lg(n+1) \rceil$,则$i_{min}=j$最多出现$2^{k-j-1}$次,总共的时间是$\sum_{j=0}^{k-1}2^{k-j-1}2^{j}=O(nlgn)$,平均每次时间$O(lgn)$。\\
而再均摊插入数组的时间,对于长度m,其最坏情况就是标号为$t=\lceil lg(m+1) \rceil-1$的数组中删除数字了。对于长度1到n,$k=\lceil lg(n+1) \rceil$,$t=j$最多出现$2^{j}$次。总共时间就是$\sum_{j=0}^{k-1}2^{j}2^{j}=O(4^{k})=O(n^{2})$,平均时间为O(n)。\\
两者时间相加,可以得出删除的均摊时间是O(n),当然,这里每次都是考虑的最坏情况,其期望的时间复杂度应该远小于这个数字。
\section{T3} 
\subsection{1}
\paragraph{算法}
如果$k\leq x$,则执行把x缩小到k的函数\\
否则先把x缩小到$-\infty$,然后提取最小的元素,最后插入新元素k\\
\paragraph{复杂度}
$k \leq x$的时候,复杂度和把x缩小到k一样,均摊下来是O(1)。\\
$k > x$的时候,把x缩小到$-\infty$均摊复杂度O(1),提取最小的元素均摊复杂度O(lgn),插入新元素均摊复杂度O(1),因此总共复杂度O(lgn)\\
\subsection{2}
\paragraph{算法}
多一个双向循环链表H.leaf存储H的所有叶子节点,同时每个节点要有两个指针,分别指向其在树链表,叶子链表里的位置。\\
执行流程:每次删除H.leaf的第一个节点,如果这个节点的父亲是空,则在H.rootlist中也删除这个节点,同时更新A等信息。否则,这个节点的父亲度数-1,并且这个节点父亲的孩子链表也移除这个节点,如果这个节点父亲的度数变为0,则将其加入H.leaf中。\\
删除完q个节点后,再更新H.n,H.min等信息。\\
\paragraph{复杂度}
先不考虑之后更新H.n,H.min等信息的时间。\\
定义势函数$\phi=k(tree(H)+degree(H))$,其中k为任意正常数,tree(H),degree(H)分别代表堆中树的个数和每个节点度数之和。在堆为空的时候,两者都是0,势函数也是0。无论如何这两者一定非负,势函数一定非负。因此堆的定义正确。\\
先计算均摊时间$c=O(q)$。因为每个删除操作都是O(1)的时间,总共删除是O(q)时间。\\
之后计算$\Delta \phi$。每次删除的节点如果父亲是空,那么tree(H)-=1。但是只删除了一个叶子节点,degree(H)不变。如果被删除的节点父亲不为空,那么tree(H)不变,而被删除的节点的父亲度数-1,degree(H)-=1。因此,每删除一个节点,$\phi$都会少k,因此$\delta \phi=-qk$。\\
而最后$\hat{c}=c+\delta \phi = O(q)-kq = O(1)$,得出如果不含之后更新信息的时间,则均摊时间是O(1)。\\
而更新H.min的时候需要遍历剩余的所有树,时间是O(lgn)。而对于任意的$q\leq n-D(n)$,都能构造出一种情况,使得没有任何树被完全删去。而$q>n-D(n)$的时候,可以保证有一种最坏情况还需要遍历$n-q$个树,因此用聚集法进行均摊分析,总时间是$(n-D(n))D(n)+\sum_{i=1}^{D(n)-1}$,因为$D(n)=O(lgn)$,因此总时间是$O(nD(n))$,均摊时间为O(D(n))。\\
综上,总共均摊时间为O(D(n)),其中删除均摊时间是O(1),更新其余信息时间O(D(n))。
\end{CJK}
\end{document}
