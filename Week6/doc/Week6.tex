\documentclass{article}
\usepackage{CJKutf8}
\usepackage{latexsym}
\usepackage{pgfplots}  
\usepackage{pgfplotstable}

\title{Week 6}
\author{软73 沈冠霖 2017013569}
\begin{document}
\begin{CJK}{UTF8}{gkai}
\maketitle
\section{T1}
\paragraph{代码}
1\qquad OPTIMAL$\_$BST(A,p,r)\\
2\qquad for i=1 to n+1\\
3\qquad \qquad $e[i,i-1]=q_{i-1}$\\
4\qquad \qquad $w[i,i-1]=q_{i-1}$\\
5\qquad for l = 1 to  n\\
6\qquad\qquad for i = 1 to n-l+1\\
7\qquad\qquad\qquad  j=i+l-1\\
8\qquad\qquad\qquad e[i,j]=$\infty$\\
9\qquad\qquad\qquad w[i,j]=w[i,j-1]+$p_{j}+q_{j}$\\
10\qquad\qquad\qquad for r = root[i,j-1] to root[i+1,j]\\
11\qquad\qquad\qquad\qquad t = e[i,r-1]+e[r+1,j]+w[i,j]\\
12\qquad\qquad\qquad\qquad if t<e[i,j]\\
13\qquad\qquad\qquad\qquad\qquad e[i,j]=t\\
14\qquad\qquad\qquad\qquad\qquad root[i,j]=r\\
15\qquad return e and root\\
只是把第三层循环改成了从root[i,j-1]到root[i+1,j]
\paragraph{证明}
首先,正确性是显然的,根据题目中的结论,$root[i,j-1] \leq root[i,j] \leq root[i+1,j]$,而在求解e[i,j]之前,已经求解过所有长度为l-1以及更短的子树的最优情况和其根了,因此必然正确\\
其次,对于时间复杂度:考虑长度为l的情况,共有e[1,l]到e[n-l+1,n]这n-l+1个情况需要求解,求解情况e[i,i+l-1]需要遍历root[i+1,i+l-1] - root[i,i+l-2] + 1次。\\
对所有n-l+1种情况的求解遍历次数求和,可以得到,处理所有长度为l的情况,需要遍历l + root[l+1,n]-root[1,l-1]次,因为$\forall 1\leq i \leq j\leq n, 1\leq root[i,j] \leq n$,因此处理长度为l的情况,需要处理时间为O(n),共有n种长度,因此时间复杂度为O($n^{2}$)。又因为这样做有$\theta (n^{2})$个子问题,因此时间复杂度也是$\Omega (n^{2})$,因此时间复杂度是$\theta(n^{2})$

\section{T3}
\subsection{3.1}
证明:假设从第一行到第i-1行($1<i \leq n$)第j列的缝隙至少有$x_{i-1}$种取值,那么以第i行任意一个元素为结尾的缝隙可能来自其左上方,正上方,右上方,即使元素在最左或者最右,也能有2$x_{i-1}$种取值。因为$x_{1} = 1$,则有$2^{m-1}\leq x_{m}$,从第一行到第m行的所求缝隙至少有$n2^{m-1}$条,因此与m成指数关系
\subsection{3.2}
\paragraph{设计思路}
子问题:假设这张图片只有前i行,以像素(i,j)为结尾的缝隙的最小破坏度V[i,j]\\
状态转移方程:V[1,j] = d[1,j],V[i,j]=min(V[i-1,j],V[i-1,j-1],V[i-1,j+1])+d[i,j](所有不满足$1 \leq i \leq m,1\leq j\leq n$的V[i,j]都定义为正无穷)\\
结果:min(V[m][j])就是最小的破坏度,可以定义一个数组previous[i][j]存储以(i,j)为最下方点的一条最优子缝隙中,(i,j)点的上一个坐标,然后回溯就能得到具体的缝隙
\paragraph{证明}
首先,每个子问题之间互相独立。因为要以(i,j)为最下方点的最优子缝隙,只需要找到以(i,j-1),(i,j),(i,j+1)为最下方点的子缝隙的最优解就行了,和具体前面缝隙如何无关。\\
其次,子问题最优解是整体最优解。假设存储的V[i,j]不是所求的最优解,那么必定可以找到另一条到(i,j)的缝隙,其破坏值更小,那么从这条缝隙按照相同的道路向下延伸,得到的最终结果必定更小,与原结果是整体最优解矛盾。\\
最终,算法需要更新mn个子问题,每个子问题最多需要3次计算,因此时间复杂度是O(mn)\\
\paragraph{代码}
1\qquad OPTIMAL$\_$Seam(d,V,Seam)\\
2\qquad for i=1 to n\\
3\qquad \qquad V[1,i]=d[1,i]\\
4\qquad \qquad previous[1,i]=0\\
5\qquad for i = 2 to  m\\
6\qquad\qquad for j = 1 to n\\
7\qquad\qquad\qquad a = V[i-1,j-1],b=V[i-1,j],c=V[i-1,j+1]\\
8\qquad\qquad\qquad if(j==1)\\
9\qquad\qquad\qquad\qquad a = $+\infty$\\
10\qquad\qquad\qquad if(j==n)\\
11\qquad\qquad\qquad\qquad c = $+\infty$\\
12\qquad\qquad\qquad $if(a<b\&\&a<c)$\\
13\qquad\qquad\qquad\qquad V[i,j]=a+d[i,j],previous[i,j]=j-1\\
14\qquad\qquad\qquad else $if(c<b\&\&c<a)$\\
15\qquad\qquad\qquad\qquad V[i,j]=c+d[i,j],previous[i,j]=j+1\\
16\qquad\qquad\qquad else\\
17\qquad\qquad\qquad\qquad V[i,j]=b+d[i,j],previous[i,j]=j\\
18\qquad min = $\infty$,minplace = 0\\
19\qquad for i = 1 to  n\\
20\qquad \qquad if(V[m,i]<min)\\
21\qquad\qquad\qquad min = V[m,i],minplace=i\\
22\qquad currentplace=minplace
23\qquad for i = m to  1\\
24\qquad \qquad Seam[i]=currentplace\\
25\qquad \qquad currentplace = previous[i,currentplace]\\
26\qquad return min and Seam\\
此时min就是最小的损耗值,Seam[i]代表第i行位于这个缝隙里的是(i,Seam[i])像素
\section{T2} 
\subsection{设计思路}
\paragraph{1.划分子问题}设原数组为A,min[i]表示长度为i的单调递增子序列长度最小末尾元素
\paragraph{2.初始值}设置min[i]全部为正无穷,当前已经更新0个元素\\
\paragraph{3.状态转移方程}
设当前更新第i个元素,$\min[j]\leq A[i] < min[j+1]$,则min[j+1]
 = A[i],更新到$i>n$结束,结果就是最大的使得min[j]不为正无穷的j\\
  想要求出最长的一个序列,可以先用一个数组sequence[i]记录min[i]对应的最小末尾的下标。在更新每个元素的时候用一个数组previous[i]记录A[i]的前一个数下标,也就是previous[i] = sequence[j],这样递推就可以求出序列\\
\paragraph{4.正确性证明}首先,min数组单调递增。因为假设已经更新了m个元素,如果有$min[i]>min[j],i<j$,则在min[j]对应的单调递增子序列里截取前i个数,他们一定是单调递增的,且第i个数$p \leq min[j] < min[i]$,与$min[i] \leq p $矛盾。\\
 其次,如果更新的第i个新元素有$\min[j]\leq A[i] < min[j+1]$,则将其放到此时min[j]对应的那个单调递增子序列后面,就可以产生一个最小末尾元素小于此时min[j+1]的长为j+1的单调递增子序列,而对于$k > j+1,min[k-1]>A[i]$,不可能找到一个在前i-1个元素的序列中长度为k-1的单调递增子序列,使得可以添加A[i]让其仍然满足条件,所以此时,min为更新i个元素后的局部最优解。因此,根据数学归纳法,所有元素全部更新完后,min为整体最优解\\
\paragraph{5.复杂度分析}每次更新只需要一个时间复杂度O(n)的二分查找,一共更新n次,所以总共时间复杂度是O(nlgn)\\
\subsection{测试结果}
\paragraph{测试环境}
CPU:Inter Core i5-6300HQ,2.3GHZ\\
内存:12G\\
环境:VS2017,release模式
\paragraph{正确性分析}
先测试3组特殊数据(完全正序,完全倒序,完全相等,再测试5组长度为5的随机数据,详细结果见表1。可以看出,测试的结果完全正确,可以初步说明结果基本正确\\
\paragraph{时间分析}
算法的理论复杂度是O(nlgn),而实际我测试了从$10^{1}$到$10^{8}$的8组数据,得到运行时间如表2。\\
可以看出,在数据规模很小$(\leq10000)$的时候,运行时间是不定波动的,因为我的代码在测试时间的时候还有I/O,这会影响时间,但是,可以看出,这个时候运算非常快。\\
在数据规模大大的时候($\geq1000000$),可以看出运行时间明显和数据规模正相关,而且数据规模每扩大10倍,时间增加10倍多一些,大致可以看出nlgn的关系\\
\begin{table}[!htbp] 
	\small
	\caption{几组数据的运行结果和预期结果对比}
	\begin{flushleft} 
		\begin{tabular}{|l|l|l|l|l|l|l|l|l|l|l|l|} 
			\hline 测量序号 & 1 & 2 & 3 & 4 & 5 & 6 & 7 & 8  \\ 
			\hline 数据 &1 2 3 4 5&5 4 3 2 1&5 5 5 5 5&41 67 3 0 69&2 7 5 6 6&5 45 81 7 61&91 95 42 27 36&9 4 2 5 9 \\ 
			\hline 预期结果
			&5&1&5&3&4&3&2&3  \\
			\hline 结果 
			&5&1&5&3&4&3&2&3  \\ 
			\hline 是否正确
			&是&是&是&是&是&是&是&是  \\ 
			\hline
		\end{tabular} 

	\end{flushleft} 
\end{table}

\begin{table}[!htbp] 
	
	\caption{不同数据规模下运行的时间}
	\begin{flushleft} 
		\begin{tabular}{|l|l|l|l|l|l|l|l|l|l|l|} 
			\hline 测量序号 & 1 & 2 & 3 & 4 & 5 & 6 & 7 & 8 \\ 
			\hline 数据范围 &10&100&1000&10000&$10^{5}$&$10^{6}$&$10^7$&$10^{8}$ \\ 
			\hline 算法运行时间 (ms)
			&2.6&8&1.8&4.4&25.4&179.8&1766.6&19872.2  \\
			\hline
		\end{tabular} 
		注:每组数据都是运行5次后取的平均值
	\end{flushleft} 
\end{table}


\pgfplotsset{}

\begin{tikzpicture}

\begin{axis}[legend pos=outer north east] % 将图例放在图外,位于图的东北角

\addplot 
table[]                         
{           		                
	lgN T
	1 2.6
	2 8
	3 1.8
	4 4.4

};

\addlegendentry{算法运行时间}         


\end{axis}
\end{tikzpicture}\\
注:横轴代表$log_{10}{n}$,对于超过100s的,其时间显示为100s

\pgfplotsset{}

\begin{tikzpicture}

\begin{axis}[legend pos=outer north east] % 将图例放在图外,位于图的东北角
\addplot 
table[]                         
{           		                
	lgN T
	5 25.4
	6 179.8
	7 1766.6
	8 19872.2

	
};


\addlegendentry{算法运行时间}         

\end{axis}
\end{tikzpicture}\\
注:横轴代表$log_{10}{n}$,对于超过100s的,其时间显示为100s



\end{CJK}
\end{document}
